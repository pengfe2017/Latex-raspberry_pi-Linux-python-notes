\documentclass[UTF8,fancyhdr,a4paper]{ctexart}

% Title page setup: The following is for titlepage setup
\title{title title title title}
\author{李鹏飞}
\date{2019.12.15}

% Margin setup: Set up margin using geometry package
\usepackage[margin = 1in]{geometry}

% Header footer setup: Fancyhdr package sets the header and footer
\usepackage{fancyhdr}
\setlength{\headheight}{40pt}% if you don't have this command, warning for \headheight is too small will appear.
\pagestyle{fancy}% also controls the toc tof tot pages

% 2nd way of setting header
%\pagestyle{myheadings} % this command conflict with the command above
%\markright {李鹏飞\hfill 业余爱好 \hfill}% with the command above, this sets the header

% Fancyhdr details setup
%\fancyhead[l]{我的Logo}% 2nd way of setting header and footer
\lhead{\includegraphics[width = 1cm]{YinYang.jpg}} % lhead = left head setup
\fancyhead[c]{文档名称}
\fancyhead[r]{文档密级}
\lfoot{12.21.2019}
\cfoot{保密信息,未经授权禁止扩散}
\rfoot{page number}
\usepackage{lastpage}% Use lastpage information
\rfoot{第{\thepage} 页, 共\pageref{LastPage}页}
\renewcommand{\footrulewidth}{1pt}% header line width control
\renewcommand{\headrulewidth}{1pt}% footer line width control

% Table of content setup
\usepackage{tocloft}
\tocloftpagestyle{fancy}% to use fancy header and footer, otherwise the tocloft package will overdrive the fancy page setup.
%\usepackage[titles]{tocloft}

\usepackage[xetex]{hyperref}
\usepackage{cite}
\usepackage{listings}
\usepackage{graphicx}
\usepackage{xcolor}


\begin{document}

%Making title page
\begin{titlepage}
\maketitle
\pagenumbering{gobble}
\centering
\vspace{10cm}
\includegraphics[width = 0.2\textwidth]{YinYang.jpg}\par
\vspace{1cm}
{\huge pengfei.li2017@outlook.com}\par
\vspace{0.5cm}
{\small 版权所有,侵权必究}\par
\vspace{0.5cm}
{\scshape \small All Rights Reserved}
\end{titlepage}

%Decorating toc page
\pagenumbering{roman} % Sets page number in toc, tot, tof
\tableofcontents
\addcontentsline{toc}{section}{目录}%note section here in the second bracket
\clearpage
\listoffigures
\addcontentsline{toc}{section}{插图}
%\addcontentsline{toc}{chapter}{List of Figures}
\clearpage
\listoftables
\addcontentsline{toc}{section}{表格}
%\addcontentsline{toc}{chapter}{List of Tables}
\clearpage

%Writing your document here
\newpage
\pagenumbering{arabic}
\section{第一个Section}

\newpage
\section{第二个Section}

\newpage
\section{第三个Section}

\newpage
\section{文献的用法}

%Bib refrences
 这是一个 refrences\cite{7398117} 的例证。不过需要注意的是以下的 code 当中:

\begin{lstlisting}[language = Python]
    \bibliography{mybib}
\end{lstlisting}
mybib 是一个名字为\space'.bib' 的 database,它必须放在和 Tex 一起的当前目录中。否则会出现,编译不通过。
并且 {\textbackslash} usepackage cite 必须要在 preamble 当中提前设置。{\texttilde}\\

注意如果 xelatex, bibtex, xelatex, xelatex,这种方式 compile tex 文档,要注意,文献的三要素都要有,否则就会出错,文献的\textcolor{red}{四要素}包括:
\begin{enumerate}
    \item \textbackslash usepackage\{cite\}, to use the cite function, you have to include cite module in the tex file.
    \item \textbackslash cite\{123456\}, here 123456 means a reference number in the .bib file.
    \item \textbackslash bibliographystyle\{plain\} 设置引用文献的个格式。
    \item \textbackslash bibliography{mybib} 读取文献源文件,注意 mybib.bib 必须在当前目录中。
\end{enumerate}

\bibliographystyle{plain}
\bibliography{mybib}

\end{document}