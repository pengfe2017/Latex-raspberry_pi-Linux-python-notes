\documentclass[UTF8,fancyhdr,a4paper]{ctexart}

% Title page setup: The following is for titlepage setup 
\title{中文latex排版用法笔记-ctex,XeTeX,Linux}
\author{李鹏飞}
\date{2019-12-15}

% Margin setup: Set up margin using geometry package
\usepackage[margin = 1in]{geometry}

% Header footer setup: Fancyhdr package sets the header and footer
\usepackage{fancyhdr}
\setlength{\headheight}{40pt}% if you don't have this command, warning for \headheight is too small will appear.
\pagestyle{fancy}% also controls the toc tof tot pages

% 2nd way of setting header
%\pagestyle{myheadings} % this command conflict with the command above
%\markright {李鹏飞\hfill 业余爱好 \hfill}% with the command above, this sets the header

% Fancyhdr details setup
%\fancyhead[l]{我的Logo}% 2nd way of setting header and footer 
\lhead{\includegraphics[width = 1cm]{YinYang.jpg}} % lhead = left head setup
\fancyhead[c]{文档名称}
\fancyhead[r]{文档密级}
\lfoot{12-21-2019}
\cfoot{保密信息,未经授权禁止扩散}
\rfoot{page number}
\usepackage{lastpage}% Use lastpage information
\rfoot{第\thepage 页,共 \pageref{LastPage} 页}
\renewcommand{\footrulewidth}{1pt}% header line width control
\renewcommand{\headrulewidth}{1pt}% footer line width control

% Table of content setup
\usepackage{tocloft}
\tocloftpagestyle{fancy}% to use fancy header and footer, otherwise the tocloft package will overdrive the fancy page setup.
%\usepackage[titles]{tocloft}

\usepackage[xetex]{hyperref}

\begin{document}

%Making title page
\begin{titlepage}
\maketitle
\pagenumbering{gobble}
\centering
\vspace{10cm}
\includegraphics[width = 0.2\textwidth]{YinYang.jpg}\par
\vspace{1cm}
{\huge pengfei.li2017@outlook.com}\par
\vspace{0.5cm}
{\small 版权所有,侵权必究}\par
\vspace{0.5cm}
{\scshape \small All Rights Reserved}
\end{titlepage}

%Decorating toc page
\pagenumbering{roman} % Sets page number in toc, tot, tof
\tableofcontents
\addcontentsline{toc}{section}{目录}%note section here in the second bracket
\clearpage
\listoffigures
\addcontentsline{toc}{section}{插图}
%\addcontentsline{toc}{chapter}{List of Figures}
\clearpage
\listoftables
\addcontentsline{toc}{section}{表格}
%\addcontentsline{toc}{chapter}{List of Tables}
\clearpage

%Writing your document here
\newpage
\pagenumbering{arabic}
\section{ctex macro 下的模板设置}

\subsection{Title page 的设置}
使用 begin titlepage 环境,对 title page 进行自定义设置,如图片的插入等;
\subsection{Header 和 Footer 的设置}
这里使用了 fancyhdr package。注意 Lastpage package 的使得 left pagefoot 的页码特特效,第 X 页,共 X 页;
\subsection{toc 或者目录的设置}
使用了 tocloft package。注意这个 package 会覆盖 fancyhdr 的功能,需要命令 tocloftpagestyle(fancy),强行保留 fancyhdr 的功能;\par
hyperef 的使用:注意在 xelatex 编辑环境下。\textbackslash usepackage [xetex] \{ hyperref \},注意 [\space] 中间的选项一定要添加 xetex。\par 
意外发现:\textbackslash[ \space ]\textbackslash ,中间竟然可以输入公式。如下:
\[ x = y^2 \]


\subsection{特殊字符的输入}

\begin{enumerate}
\item
字符输入:一般使用 \textbackslash \space 作为前导。比如 \{ 的输入:Latex 命令为,\textbackslash \{。

\item 
\S \space dollar sign 的使用: 成对使用,用于输入公式,例子如下:
Inline equations input: \$some equations\$, e.g. $\partial T^2/\partial x^2 + \partial T^2/\partial y^2 =\frac{k}{\rho*c_p}*{\partial T/\partial t} $

\item
\textasciitilde 符号的输入有些特殊,必须加入 ascii 关键词:\textbackslash textasciitilde

\end{enumerate}

\subsection{字体的基本设置}



%\addcontentsline{toc}{section}{目录} 这个命令将 toc,tof,tot都加入了目录页中。
\subsection{页码的设置}
使用命令:\textbackslash pagenumbering \{ gobble \}

\newpage
\section{第二个Section}

\subsection{接下来要做的事情}
\begin{enumerate}
\item
URL 的输入: 对于 XeLatex + View PDF 设置,在 preamble 当中使用:\\
\textbackslash usepackage[xetex]\{hyperref\},在具体的语句使用例如:\\ \url{https://www.lifewire.com/uses-of-linux-command-find-2201100}
\item
Texworks 的 Linux 安装
\item
Bibliography 的引入
\item
图片的插入及基本设计
\item
表格的插入及基本设置



\end{enumerate}

\newpage
\section{第三个Section}

\newpage
\section{第四个Section}



\end{document}