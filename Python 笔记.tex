\documentclass[UTF8,fancyhdr,a4paper]{ctexart}

% Title page setup: The following is for titlepage setup 
\title{Python programming notes}
\author{李鹏飞}
\date{2020-01-04}

% Margin setup: Set up margin using geometry package
\usepackage[margin = 1in]{geometry}

% Header footer setup: Fancyhdr package sets the header and footer
\usepackage{fancyhdr}
\setlength{\headheight}{40pt}% if you don't have this command, warning for \headheight is too small will appear.
\pagestyle{fancy}% also controls the toc tof tot pages

% 2nd way of setting header
%\pagestyle{myheadings} % this command conflict with the command above
%\markright {李鹏飞\hfill 业余爱好 \hfill}% with the command above, this sets the header

% Fancyhdr details setup
%\fancyhead[l]{我的Logo}% 2nd way of setting header and footer 
\lhead{\includegraphics[width = 1cm]{YinYang.jpg}} 
% lhead = left head setup
\fancyhead[c]{文档名称}
\fancyhead[r]{文档密级}
\lfoot{12-21-2019}
\cfoot{保密信息,未经授权禁止扩散}
\rfoot{page number}
\usepackage{lastpage}% Use lastpage information
\rfoot{第\thepage 页,共 \pageref{LastPage} 页}
\renewcommand{\footrulewidth}{1pt}% header line width control
\renewcommand{\headrulewidth}{1pt}% footer line width control

% Table of content setup
\usepackage{tocloft}
\tocloftpagestyle{fancy}% to use fancy header and footer, otherwise the tocloft package will overdrive the fancy page setup.
%\usepackage[titles]{tocloft}

\usepackage[xetex]{hyperref}
\usepackage{xcolor}
\newcommand{\pflred}[1]{\textcolor{red}{#1}}

\usepackage{listings}

\begin{document}

%Making title page
\begin{titlepage}
\maketitle
\pagenumbering{gobble}
\centering
\vspace{10cm}
\includegraphics[width = 0.2\textwidth]{YinYang.jpg}\par
\vspace{1cm}
{\huge pengfei.li2017@outlook.com}\par
\vspace{0.5cm}
{\small 版权所有,侵权必究}\par
\vspace{0.5cm}
{\scshape \small All Rights Reserved}
\end{titlepage}

%Decorating toc page
\pagenumbering{roman} % Sets page number in toc, tot, tof
\tableofcontents
\addcontentsline{toc}{section}{目录}%note section here in the second bracket
\clearpage
\listoffigures
\addcontentsline{toc}{section}{插图}
%\addcontentsline{toc}{chapter}{List of Figures}
\clearpage
\listoftables
\addcontentsline{toc}{section}{表格}
%\addcontentsline{toc}{chapter}{List of Tables}
\clearpage

%Writing your document here
\newpage
\pagenumbering{arabic}
%————————————————————————————————————————————————————————————主要内容
\section{Package installation}
\subsection{pip install using mirror sites}
using command: \textit{pip install -i https://pypi.tuna.tsinghua.edu.cn/simple some-package}\par
or set the mirror permanently using:\\
pip install pip -U\\
pip config set global.index-url https://pypi.tuna.tsinghua.edu.cn/simple
\subsection{alias 别名}
有些情况下,在 linux 系统中,你会发现 pip 安装了某个 module,结果直接在 terminal 运行该 module 命令,却没有任何反映。这时运行 python3 -m IPython 却可以运行,这是因为 IPython 没有注册到系统中,需要运行,alias ipython = “python3 -m IPython”
%————————————————————————————————————————————————————————————主要内容
\section{Python 图像处理}
\subsection{20200104 Image2ExcelData}

%————————————————————————————————————————————————————————————IPython
\newpage
\section{IPython}
\subsection{Useful commands}
execute commands in IPython like in window cmd or linux Terminal.
\begin{itemize}
\item
\%cd:改变当前的工作目录
\item
\%edit:打开编辑器,并关闭编辑器后执行键入的代码
\item
\%env:显示当前环境变量
\item
\%pip install [pkgs]:无需离开交互式shell,就可以安装软件包
\item
\%time 和 \%timeit:测量执行Python代码的时间
\end{itemize}

%————————————————————————————————————————————————————————————Django
\newpage
\section{Django}
\subsection{Django html 路径}
\begin{lstlisting}[language = python]
path('', include("mainpage.urls")),
\end{lstlisting}
这里网站的 working directory 是 mainpage app 下的 folder.

设置 css 或者图片等 static file 注意事项:\\
\begin{enumerate}
\item 在 app folder 下创建 static 文件夹,如同 templates 一样。
\item 在 static 文件下创建与 app 同名的 appname 文件夹,将所有的 css,img 等放入文件夹中。
\item html template 中 写入以下 code:
\begin{lstlisting}[language = python] 
, 
具体语句中,写入 src or href 
= “”
\end{lstlisting}
注意这里的 appname 必须要写,这是用以区分,不同 app 使用的不同的 static files。Django 非常不错的地方,就在与如此,Django 可以帮助用户寻找相应的 app 下的 static files。

\end{enumerate}





\end{document}