\documentclass[UTF8,fancyhdr,a4paper]{ctexart}

% Title page setup: The following is for titlepage setup 
\title{Pytorch and Machine Learning}
\author{李鹏飞}
\date{2020-02-01}

% Margin setup: Set up margin using geometry package
\usepackage[margin = 1in]{geometry}

% Header footer setup: Fancyhdr package sets the header and footer
\usepackage{fancyhdr}
\setlength{\headheight}{40pt}% if you don't have this command, warning for \headheight is too small will appear.
\pagestyle{fancy}% also controls the toc tof tot pages

% 2nd way of setting header
%\pagestyle{myheadings} % this command conflict with the command above
%\markright {李鹏飞\hfill 业余爱好 \hfill}% with the command above, this sets the header

% Fancyhdr details setup
%\fancyhead[l]{我的Logo}% 2nd way of setting header and footer 
\lhead{\includegraphics[width = 1cm]{YinYang.jpg}} % lhead = left head setup
\fancyhead[c]{文档名称}
\fancyhead[r]{文档密级}
\lfoot{12-21-2019}
\cfoot{保密信息,未经授权禁止扩散}
\rfoot{page number}
\usepackage{lastpage}% Use lastpage information
\rfoot{第\thepage 页,共 \pageref{LastPage} 页}
\renewcommand{\footrulewidth}{1pt}% header line width control
\renewcommand{\headrulewidth}{1pt}% footer line width control

% Table of content setup
\usepackage{tocloft}
\tocloftpagestyle{fancy}% to use fancy header and footer, otherwise the tocloft package will overdrive the fancy page setup.
%\usepackage[titles]{tocloft}
%===============================================================
\usepackage[xetex]{hyperref}
\usepackage{xcolor}
%===============================================================

\begin{document}

%Making title page
\begin{titlepage}
\maketitle
\pagenumbering{gobble}
\centering
\vspace{10cm}
\includegraphics[width = 0.2\textwidth]{YinYang.jpg}\par
\vspace{1cm}
{\huge pengfei.li2017@outlook.com}\par
\vspace{0.5cm}
{\small 版权所有,侵权必究}\par
\vspace{0.5cm}
{\scshape \small All Rights Reserved}
\end{titlepage}

%Decorating toc page
\pagenumbering{roman} % Sets page number in toc, tot, tof
\tableofcontents
\addcontentsline{toc}{section}{目录}%note section here in the second bracket
\clearpage
\listoffigures
\addcontentsline{toc}{section}{插图}
%\addcontentsline{toc}{chapter}{List of Figures}
\clearpage
\listoftables
\addcontentsline{toc}{section}{表格}
%\addcontentsline{toc}{chapter}{List of Tables}
\clearpage

%Writing your document here
\newpage
\pagenumbering{arabic}
\section{Pytorch 安装}
在中国镜像要选清华,否则速度会非常的慢。
\subsection{清华镜像设置命令}
\begin{enumerate}
\item conda config --add channels https://mirrors.tuna.tsinghua.edu.cn/anaconda/cloud/pytorch (设置 pytorch 的镜像,注意 \textcolor{red}{pytorch 镜像和 conda 镜像不是一个地方。} )
\item conda config --add channels https://mirrors.tuna.tsinghua.edu.cn/anaconda/pkgs/free/ (conda 自己的镜像)
\item conda config --set show\_channel\_urls yes
\item conda install pytorch torchvision cudatoolkit=10.1 -c pytorch (pytorch 按照相应环境生成的 conda 安装命令,注意,如果使用清华镜像,不必加 \textcolor{red}{-c pytorch})
\item reference: \url{https://blog.csdn.net/xd_wjc/article/details/80587488}

\end{enumerate}

\newpage
\section{Good References}
\begin{enumerate}
\item towards data science: \url{https://towardsdatascience.com}
\item machine learning mastery: \url{https://machinelearningmastery.com}

\end{enumerate}

\newpage
\section{第三个Section}

\newpage
\section{第四个Section}



\end{document}