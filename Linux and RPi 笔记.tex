\documentclass[UTF8,fancyhdr,a4paper]{ctexart}

% Title page setup: The following is for titlepage setup 
\title{Linux and RPi notes}
\author{李鹏飞}
\date{2020-01-01}

% Margin setup: Set up margin using geometry package
\usepackage[margin = 1in]{geometry}

% Header footer setup: Fancyhdr package sets the header and footer
\usepackage{fancyhdr}
\setlength{\headheight}{40pt}% if you don't have this command, warning for \headheight is too small will appear.
\pagestyle{fancy}% also controls the toc tof tot pages

% 2nd way of setting header
%\pagestyle{myheadings} % this command conflict with the command above
%\markright {李鹏飞\hfill 业余爱好 \hfill}% with the command above, this sets the header

% Fancyhdr details setup
%\fancyhead[l]{我的Logo}% 2nd way of setting header and footer 
\lhead{\includegraphics[width = 1cm]{YinYang.jpg}} % lhead = left head setup
\fancyhead[c]{文档名称}
\fancyhead[r]{文档密级}
\lfoot{12-21-2019}
\cfoot{保密信息,未经授权禁止扩散}
\rfoot{page number}
\usepackage{lastpage}% Use lastpage information
\rfoot{第\thepage 页,共 \pageref{LastPage} 页}
\renewcommand{\footrulewidth}{1pt}% header line width control
\renewcommand{\headrulewidth}{1pt}% footer line width control

% Table of content setup
\usepackage{tocloft}
\tocloftpagestyle{fancy}% to use fancy header and footer, otherwise the tocloft package will overdrive the fancy page setup.
%\usepackage[titles]{tocloft}

\usepackage[xetex]{hyperref}
\usepackage{xcolor}
\newcommand{\pflred}[1]{\textcolor{red}{#1}}


%============================================= packages

\PassOptionsToPackage{hyphens}{url}
\usepackage[xetex]{hyperref}
%\usepackage[hyphenbreaks]{breakurl}
\usepackage[hyphens]{url}
\usepackage{cite}


%============================================= packages




\begin{document}

%Making title page
\begin{titlepage}
      \maketitle
      \pagenumbering{gobble}
      \centering
      \vspace{10cm}
      \includegraphics[width = 0.2\textwidth]{YinYang.jpg}\par
      \vspace{1cm}
      {\huge pengfei.li2017@outlook.com}\par
      \vspace{0.5cm}
      {\small 版权所有,侵权必究}\par
      \vspace{0.5cm}
      {\scshape \small All Rights Reserved}
\end{titlepage}

%Decorating toc page
\pagenumbering{roman} % Sets page number in toc, tot, tof
\tableofcontents
\addcontentsline{toc}{section}{目录}%note section here in the second bracket
\clearpage
\listoffigures
\addcontentsline{toc}{section}{插图}
%\addcontentsline{toc}{chapter}{List of Figures}
\clearpage
\listoftables
\addcontentsline{toc}{section}{表格}
%\addcontentsline{toc}{chapter}{List of Tables}
\clearpage




%Writing your document here
\newpage
\pagenumbering{arabic}
\section{Linux/Elementary OS 在 Nvidia 安装之后的急救}
\begin{enumerate}
      \item 安装完 Nvidia 显卡 driver 之后发现,系统无法正常启动,且在一个 command window 下反复循环。
      \item google 寻找了相关措施,也不能够解决问题。
      \item ctr alt F1 反复按压也结局不了问题。
      \item F12 is for booting bios for my Huawei Laptop.
      \item 胡乱按了一些 F1 键盘,终于进入 console,执行 sudo apt-get purge nvidia*
\end{enumerate}

20200625, second time, I am trying to install nvidia driver, this time, I got the same infinite loop problem. \pflred{never ever try sudo apt install nvidia-340 again, this is a pretty older driver, it does not support my huawei nvidia driver.} Faulty behaviors and solutions are listed below:
\begin{enumerate}
      \item run nvidia-smi command, this is a command normally for checking if the driver is installed correctly or not.
      \item system reminds you to install these two packages, nvidia-340 and nvidia-utils-390, then I did run sudo apt install nvidia-340, and then it stuck.
      \item try to follow the aforementioned tips by keeping pressing the F1 key, it doesn't work.
      \item Esc helps gets into Debian window, also called grub mode, like recovery mode in windows? Try to google Ubuntu RecoveryMode, it shows detailed instruction to booting into recovery mode.
      \item \pflred{Esc key is the critical button} to get into recovery mode. However, we have to check what is this UEFI, because for BIOS, you enter grub mode by pressing shift key.
      \item \url{https://itsfoss.com/check-uefi-or-bios/}, here it gives a clear explanation of UEFI and BIOS mode. 
      \item After you entered the grub mode, then select an option to enter into the pure console mode, and purge all nvidia setup.
\end{enumerate}





\section{Linux common commands}

\subsection{Notes on 20200101}

\subsubsection{check ip addresses of all users in local subnet}

sudo nmap -sn 192.168.0.0/24\\
results:\\
Starting Nmap 7.60 ( https://nmap.org ) at 2020-01-01 17:14 CST\\
Nmap scan report for \_ gateway (192.168.0.1)\\
Host is up (0.0043s latency).\\
MAC Address: C8:3A:35:2B:6C:38 (Tenda Technology)\\
Nmap scan report for 192.168.0.100\\
Host is up (0.0040s latency).\\
MAC Address: 94:9F:3E:C3:E1:BD (Sonos)\\
Nmap scan report for 192.168.0.101
Host is up (0.19s latency).\\
MAC Address: 9C:2E:A1:2E:D8:84 (Unknown)\\
Nmap scan report for 192.168.0.103\\
Host is up (-0.053s latency).\\
MAC Address: F4:63:1F:F6:66:43 (Unknown)\\
Nmap scan report for 192.168.0.108\\
Host is up (0.076s latency).\\
MAC Address: BC:83:85:C7:27:73 (Microsoft)\\
Nmap scan report for 192.168.0.109\\
Host is up (0.076s latency).\\
MAC Address: B8:27:EB:26:81:1F (Raspberry Pi Foundation)\\
Nmap scan report for 192.168.0.110\\
Host is up (0.076s latency).\\
MAC Address: A0:51:0B:87:5E:C4 (Unknown)\\
Nmap scan report for pengfei-KLV-WX9 (192.168.0.107)\\
Host is up.\\
Nmap done: 256 IP addresses (8 hosts up) scanned in 4.61 seconds\\

\subsection{Linux remote desktop windows10 home edition}
windows10 home edition 不允许 remote desktop,虽然系统有这个选项询问你是否允许 remote desktop,但是在另一个地方又告诉你 home edition 是没有这个功能的。
以下是解决方案:
\begin{enumerate}
      \item 使用 TightVNC:
            网址:\url{https://www.tightvnc.com/download.php}
            在 windows10 直接安装 TightVNC,一路安装不用改任何东西,安装完毕。设置require authentication,设置密码。\\
            \pflred
            {TightVNC 的一些问题:
                  \begin{enumerate}
                        \item VNC 有些时候不稳定,需要重启系统,有可能是 VPN 的问题。
                        \item VNC 用 laptop 的 touch pad 右键不好使,只能用鼠标。
                  \end{enumerate}
            }



      \item 使用 RDPwrapper 在 github 下又下载:\\
            不推荐这个使用方法,CSDN 上很多人反映不可用,而且下载的时候 chrome 会报错。另外还有修改注册表的,不推荐使用。
      \item 在 linux 系统下安装 remmina:\\
            sudo apt-get install remmina remmina-plugin-*\\
            注意尽量需找 apt-get 支持的语句,snap 和 dnf 都不好用,估计是支持其他 linux distro 的。




\end{enumerate}

\subsection{Windows10 remote linux}
需要注意的是我的 raspberryPi 直接允许 VNC 即可,直接通过正常的 windows remote desktop 即可远程登录。但是elementory os 不具有这个功能,于是需要安装。
\begin{enumerate}
      \item 在 linux 上直接安装 xrpd:sudo apt-get install xrdp。
      \item 直接登录,出现内部错误:这个是由于 VPN 的作用,关掉在开启即可。
      \item 登录之后发现闪退:这是由于 linux 没有安装 xfce4。\\ sudo apt-get install xfce4。\par
            百度搜索,通过 windows RDP 连接 Linux。

            \pflred{如果这时候出现所说的Ubuntu安中装的xrdp,桌面共享设置为允许,虚拟机设置的是桥接,虚拟机内外可以互相ping 通;
                  在Windows下用 mstsc 连接Ubuntu桌面,跳出账号密码页面,登录后自动退出(闪退)的解决方法:sudo apt-get install xfce4
                  echo xfce4-session > \~/.xsession
                  touch .session
                  sudo vim /etc/xrdp/startwm.sh
                  在. /etc/X11/Xsession前面加
                  xfce4-session
                  然后重启 sudo service xrdp restart。}


\end{enumerate}


\subsection{查找文档命令 find}

find + [target directory] + [options] + [file or folder name]:
\begin{enumerate}
      \item
            target directory:
            Period (.): Specifies the current and all nested folders.
            Forward Slash (/): Specifies the entire filesystem.
            Tilde (\~): Specifies the active user's home directory.
      \item
            options:
            -name or -atime n
      \item
            file or folder name:
            * start sign represents wild card.
      \item one example:
            find . -name *.tex
      \item another example:
            find /home -name wsgi.py
      \item
            reference:
            \url	{https://www.lifewire.com/uses-of-linux-command-find-2201100}


\end{enumerate}

\subsection{Copy file}

The best way I found so far is to use: sftp pi@192.168.0.109, then using the same login password as in: ssh pi@192.168.0.109.
Some common commands are listed below:
\begin{enumerate}
      \item ls: ls list the remote directory. lls is for the local.
      \item pwd: present working directory. lpwd: local present working directory.
      \item get: copy a file into the current local working directory.

\end{enumerate}





\newpage

\section{Linux Screen APP 用法}
\pflred{Screen 是一个非常重要的 APP,可以在 Session/Terminal 关闭的时候,仍然在后台运行},相当于我以前使用的命令:\textit{nohup python3 xx.py \&}
\subsection{键盘命令(直接 prcess 键盘) ctl A + key 系列}
例如:ctr A + k,是 kill 当前的命令。具体如下:
\begin{enumerate}
      \item ctr A + k: kill the current screen.
      \item ctr A + ?: help window for ctr A commands.
      \item ctr A + [ or escapekey: start copy mode, you can use arrow key to move up and down, if you press enter key, they you are selecting something for copy. If you don't want any copy, press escape key again.
      \item ctr A + ]: paste the selected things.
      \item reference: google scroll inside screen or pause output.
\end{enumerate}
\subsection{Console 命令 (在 console 当中输入) screen -[key]}
例如:screen -ls:list all screens active.具体如下:
\begin{enumerate}
      \item screen -list: = screen -ls
      \item screen -S screenname: create a screen with a name "screenname"
      \item screen: create a screen without a specified name, but the system will automatically assign a name to the screen.
      \item screen -d screenid/screenname: detach a screen locally or remotely. If you have a session running locally, you have to start a new terminal somewhere else to detach it.
      \item screen -r screenid/screenname: reattach a screen.
      \item reference: google 15 linux screen command for dealing terminal sessions.
\end{enumerate}


\section{Linux VPN 用法}
\cite{0040s}
本文具体采用西游 VPN \url{xiyou360.net}.
\begin{enumerate}
      \item 下载并解压 Ciso AnyConnect
      \item 解压文档
      \item cd 到指定文件夹,执行 cd vpn
      \item 执行 sudo .\textbackslash vpn\textunderscore install.sh
\end{enumerate}

\section{ Linux firewall iptables}
see the reference : \begin{verbatim}
      https://www.howtogeek.com/177621/the-beginners-guide-to-iptables-the-linux-firewall/     
\end{verbatim}




\section{Raspberry Pi: Pi Model 3B+ V1.3}
\subsection{Check hostname and other information}
\textcolor{red}{hostnamectl:}\\
Static hostname: pengfei-KLV-WX9\\
Icon name: computer-laptop\\
Chassis: laptop\\
Machine ID: 82aadb731d60458194911154ce5fef65\\
Boot ID: 4405f29ae34248918f4b9b3b6d387a8b\\
Operating System: elementary OS 5.0 Juno\\
Kernel: Linux 4.15.0-70-generic\\
Architecture: x86-64\\
\textcolor{red}{whoami:} this command is to check my username.
\subsection{Login to pi with ssh}
Using command: ssh pi@IPAddress, here pi is your user name, you can check ipaddress with command, \textit{hostname -I}. ssh is in small case.
\subsection{Functions and commands}
\begin{enumerate}
      \item check Raspberry Pi information: \textit{pinout}
\end{enumerate}
\subsection{Transfer file to and from pi (or Linux system)}
Detailed information can be referenced to \url{https://www.raspberrypi.org/documentation/remote-access/}. 具体来说有一下几种形式:
\begin{enumerate}
      \item Between Linux and Windows: SFTP,SSH File Transfer Protocal. 本质来说人然是 SSH 服务,Linux 系统需要开启 SSH 服务。\textcolor{red}{Windows 系统安装 WinSCP}。
      \item Between Linux and Linux: 同上 SFTP,\textcolor{red}{不过这时候推荐 FileZilla}.
      \item Between Linux and Linux: 除了上述 SFTP 之外,还可以使用 SCP,Secure Copy 来使用。比如:scp myfile.txt pi@192.168.0.100:(这里是 destination directory, 该语句代表 /home/pi,如果是其他 directory,只需要在:后边加上制定的 directory 即可。) Copying from remote directory to local directory: scp pi@192.168.0.100:myfile.txt.
\end{enumerate}

\subsection{RPI 搭配 Django 网站框架问题及解决方案}
\begin{enumerate}
      \item \emph{Importing the numpy c-extensions failed. and Original error was: libf77blas.so.3: cannot open shared object file: No such file or directory.} 错误解决方案:\textcolor{red}{sudo apt-get install libatlas-base-dev} 。原因很可能是 RPI 系统中缺少 libff77blas.so.3 的系统文件,我是通过 google 得出的这个解决方案。
\end{enumerate}






\newpage
\section{Git file management}
\subsection{将 Local folder push 到 git repo}
一般来说,git repo 在网站上直接创建,然后在 local directory 当中使用 git clone 命令进行下载,如果不存在 git cloud 端的 directory 怎么办呢?\\
参照:\begin{sloppypar}\url{https://help.github.com/en/github/importing-your-projects-to-github/adding-an-existing-project-to-github-using-the-command-line}\end{sloppypar}
\begin{enumerate}
      \item Create a new repository. \textcolor{blue}{这一步是必须的,如果 git 网站没有这个 repository,push 会出现各种错误。}
      \item Locally git init.
      \item git add . and git commit
      \item git remote add origin remote (repo URL, this URL can be SSH or HTTP, SSH will be better, it won't ask your credentials everytime.)
      \item git push \textcolor{red}{-u} origin master. (This -u option must be there, otherwise, git pull will ask you which branch you want to merge with.)
\end{enumerate}


\newpage
\section{vscode IDE 的一些用法}
Technically, vs code is not an IDE, integrated development environment, it is an editor just like notepad but with many extensions.
Figure \ref{integrated shell} shows how to integrate shell into vscode.
\begin{figure}[h]
      \centering
      \includegraphics[width = 14cm]{Figures/inheritantshell.png}
      \caption{integrated shell}
      \label{integrated shell}
\end{figure}
\subsection{vscode upgrade}

\section{Update modules like vscode}
To update a module, e.g. vscode, you can directly use sudo apt-get, as shown in Figure\ref{Update Vscode}.
\begin{figure}[h]
      \centering
      \includegraphics[width = 14cm]{Figures/UpdateVscode.png}
      \caption{Update Vscode}
      \label{Update Vscode}
\end{figure}

\section{第二个Section}

\subsection{接下来要做的事情}
\begin{enumerate}
      \item plot to a digital buffer for Django app, so that you don't have to save a figure and load it. This can potentially solve the problem of concurrency, which means multiple users can access the app at the same time, and generate the image at the same time, but load it at different times, these two users can get an image that is created by the other one.
      \item Pyinstaller has upgraded to a new version, I should try it.


\end{enumerate}

\newpage
\section{第三个Section}

\newpage
\section{文献}
\bibliographystyle{plain}
\bibliography{mybib}

\end{document}
